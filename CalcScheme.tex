\documentclass[11pt, oneside]{amsart}   	% use "amsart" instead of "article" for AMSLaTeX format
\usepackage[top=1.2in, bottom=1.0in, left=1in, right=1in]{geometry}                		% See geometry.pdf to learn the layout options. There are lots.
\geometry{letterpaper}                   		% ... or a4paper or a5paper or ... 
%\geometry{landscape}                		% Activate for for rotated page geometry
\usepackage[parfill]{parskip}    		% Activate to begin paragraphs with an empty line rather than an indent
\usepackage{graphicx}				% Use pdf, png, jpg, or eps� with pdflatex; use eps in DVI mode
								% TeX will automatically convert eps --> pdf in pdflatex		
\usepackage{amssymb, amsmath, amsthm}

\usepackage[colorlinks=true]{hyper ref}


\newcommand{\bi}{\begin{itemize}}
\newcommand{\ei}{\end{itemize}}
\newcommand{\be}{\begin{enumerate}}
\newcommand{\ee}{\end{enumerate}}



\title[Common Syllabus Guidelines]{Common Syllabus Guidelines \\ For All Calculus Courses \\ Including Math 200, 201, 202, 262, 272}
\author{Leah Berman, Gordon Williams, Ed Bueler, Jill Faudree}
\date{\today}							% Activate to display a given date or no date

\begin{document}
\maketitle
%\section{}
%\subsection{}

\thispagestyle{empty}

Across the multiple sections of Math 200 Calculus I, Math 201 Calculus II, Math 202 Calculus III, Math 262 Calculus for Business and Economics, Math 272 Calculus for the Life Sciences, delivered on-ground, online synchronously, or online asynchronously, we suggest that all syllabi should satisfy the following requirements.

\begin{enumerate}

\item General Guidelines
\bi
\item Syllabi must \emph{clearly} satisfy university requirements.  See

\href{http://www.uaf.edu/uafgov/faculty-senate/curriculum/course-degree-procedures-/uaf-syllabus-requirements/}{\small \texttt{www.uaf.edu/uafgov/faculty-senate/curriculum/\\ \phantom{foobar} course-degree-procedures-/uaf-syllabus-requirements}}
\ei
\item Content
\bi
\item For Math 200, 201, 202, the  textbook adopted by the department for the calculus sequence should be used. All of the required (non-optional) sections listed in the department core topics list (\href{http://www.uaf.edu/dms/core/LarsonEdwardsSyllabi.pdf}{\small\texttt{www.uaf.edu/dms/core/LarsonEdwardsSyllabi.pdf}}) should be covered. 
\item Content covered in Math 262, 272 should reflect the applications-driven nature of the course.
\ei
\item Assessments
\bi
\item Exams:
\bi
\item there should be at least two exams during the semester
\item most exams should be timed
\item exams should be majority written answer (not multiple choice)
\item exams should be proctored
\ei
\item Final exam:
\bi
\item should be cumulative
\item should be representative
\bi
\item For example, a final exam for Calculus I should not simply consist of ``find the derivative'' questions.
\item The final exam in Math 262, 272 should contain some application-driven word problems.
\ei
\ei
\item Other assessed work
\bi
\item there should be regular human-graded assessments (i.e., students should be assessed and receive feedback at times other than the exams)
\item at least weekly human feedback (e.g., not just WebAssign as feedback)
\item there should be human feedback prior to the first exam
\ei
\ei
\item Contributions of assessments
\bi
\item computer-graded assessments (e.g, WebAssign, Aleks) should count for {\bf at most}  10\% of the final grade
\item exams (mid-semester and final) should contribute {\bf at least} 60\% of the final grade
\item the final exam should contribute {\bf at least} 20\% of the final grade
\item regular, non-exam assessments (e.g., quizzes, online homework, written homework) should contribute {\bf at least} 15\% of the final grade

%\item there should be least two (one?) proctored mid-semester exams, and also a proctored final exam, whose total contribution is at least 60\% (50\%?) of the final grade
\ei 

\end{enumerate}

\end{document}  